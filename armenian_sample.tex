%NOTE! compile with XeLaTeX
\documentclass[a4paper]{article}

%Հայերենի համար անհրաժեշտ բաղադրամասեր
\usepackage{fontspec}
\usepackage{polyglossia}
\setdefaultlanguage{armenian}
\newfontfamily\armenianfont{GHEA Mariam}
%\newfontfamily\armenianfont{GHEA Grapalat}
%\newfontfamily\armenianfont{Arial Unicode}

\sloppy
\renewcommand{\contentsname}{Բովանդակություն} %որ contents-ի փոխարեն հայերեն գրի
\renewcommand{\refname}{Գրականություն} %որ references-ի փոխարեն հայերեն գրի

\usepackage{indentfirst} %էս էլ որ ամեն պարագրաֆ մի մատ խորքից գրի՝ հայերեն կանոներին համապատասխան


\usepackage{amsmath}

\title{Հայերեն Լատեխ}
\author{Հայկ Սարգյան}
\date{Ապրիլ 2016}




\begin{document}
\maketitle
\tableofcontents

\section{Բաժին առաջին}
\(\zeta_H(x;p)\)-ն Հուրվիցի զետա ֆունկցիան է, որը սահմանվում է հետևյալ կերպ \cite{book:elizalde_zeta}`
%
\begin{equation}
\zeta_H(x;p)=\sum_{n=0}^{\infty}\frac{1}{(n+p)^x} \quad 0<p<1 \\
\end{equation}
%
%
\section{Բաժին երկրորդ}
%
Հուրվիցի զետա ֆունկցիան երբեմն սահմանում են նաև այսպես՝
%
\begin{equation}
\zeta_H(x;p)=\sum_{n=1}^{\infty}\frac{1}{(n+p)^x} \quad 0<p<1 \\
\end{equation}
%

\begin{thebibliography}{1}
\bibitem{book:elizalde_zeta} E. Elizalde, S. D. Odintsov, A. Romeo, A. A. Bytsenko, and S. Zerbini, \emph{Zeta Regularization Techniques with Applications} (World Scientific, Singapore, 1994).


\end{thebibliography}

\end{document}
