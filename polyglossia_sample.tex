%NOTE! compile with XeLaTeX
\documentclass[a4paper]{article}

%Հայերենի համար անհրաժեշտ բաղադրամասեր
\usepackage{fontspec}
\usepackage{polyglossia}
\setdefaultlanguage[numerals=arabic]{armenian}
\newfontfamily\armenianfont{GHEA Mariam}
%\newfontfamily\armenianfont{GHEA Grapalat}
%\newfontfamily\armenianfont{Arial Unicode}

\sloppy
\renewcommand{\contentsname}{Բովանդակություն} %որ contents-ի փոխարեն հայերեն գրի
\renewcommand{\refname}{Գրականություն} %որ references-ի փոխարեն հայերեն գրի

\usepackage{indentfirst} %էս էլ որ ամեն պարագրաֆ մի մատ խորքից գրի՝ հայերեն կանոներին համապատասխան


\usepackage{amsmath}

\title{Հայերեն Լատեխ}
\author{Հայկ Սարգյան}
\date{Հունվար 2024}




\begin{document}
\maketitle
\tableofcontents

\section{Ռիմանի զետա ֆունկցիա}
Ռիմանի զետա ֆունկցիան\footnote{Երբեմն անվանում են նաև Էյլեր-Ռիմանի զետա ֆունկցիա} կոմպլեքս փոփոխականի ֆունկցիա է, որը սահմանվում է հետևյալ կերպ՝
%
\begin{equation}
\sum_{n=1}^\infty \frac{1}{n^s} = \frac{1}{1^s} + \frac{1}{2^s} + \frac{1}{3^s} + \cdots,
\end{equation}
%
%
երբ \(\operatorname{Re}(s) > 1\), և այս շարքի անալիտիկ շարունակությամբ, երբ \(\operatorname{Re}(s) \leq 1\).  Ռիմանի զետա ֆունկցիան սովորաբար նշանակում են \(\zeta(s)\).

\section{Հուրվիցի զետա ֆունկցիա}
Հուրվիցի զետա ֆունկցիան Ռիմանի զետա ֆունկցիայի ընդհանրացում է։ Այն նշանակում են \(\zeta_H(s;p)\)։ Գրականության մեջ կարելի է հանդիպել իրարից մի փոքր տարբեր երկու սահմանումների։

\subsection{Սահմանում}
Հուրվիցի զետա ֆունկցիան սովորաբար սահմանվում է հետևյալ շարքի միջոցով \cite{book:elizalde_zeta}`
%
\begin{equation}
\sum_{n=0}^{\infty}\frac{1}{(n+p)^s} \quad p \ne 0, -1, -2, \dots \\
\end{equation}
%
%
Նկատենք, որ \(\zeta(s) = \zeta_H(s;1)\)

\subsection{Փոքր ինչ տարբեր սահմանում}
%
Հուրվիցի զետա ֆունկցիան երբեմն սահմանում են նաև հետևյալ շարքի միջոցով՝
%
\begin{equation}
\sum_{n=1}^{\infty}\frac{1}{(n+p)^s} \quad p \ne -1, -2, \dots \\
\end{equation}
%
%
Նկատենք, որ այս դեպքում \(\zeta(s) = \zeta_H(s;0)\)

\begin{thebibliography}{1}
\bibitem{book:elizalde_zeta} E. Elizalde, S. D. Odintsov, A. Romeo, A. A. Bytsenko, and S. Zerbini, \emph{Zeta Regularization Techniques with Applications} (World Scientific, Singapore, 1994).


\end{thebibliography}

\end{document}
